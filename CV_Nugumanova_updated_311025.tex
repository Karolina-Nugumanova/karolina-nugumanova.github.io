% Options for packages loaded elsewhere
\PassOptionsToPackage{unicode}{hyperref}
\PassOptionsToPackage{hyphens}{url}
\PassOptionsToPackage{dvipsnames,svgnames,x11names}{xcolor}
\documentclass[
]{article}
\usepackage{xcolor}
\usepackage[margin=1in]{geometry}
\usepackage{amsmath,amssymb}
\setcounter{secnumdepth}{-\maxdimen} % remove section numbering
\usepackage{iftex}
\ifPDFTeX
  \usepackage[T1]{fontenc}
  \usepackage[utf8]{inputenc}
  \usepackage{textcomp} % provide euro and other symbols
\else % if luatex or xetex
  \usepackage{unicode-math} % this also loads fontspec
  \defaultfontfeatures{Scale=MatchLowercase}
  \defaultfontfeatures[\rmfamily]{Ligatures=TeX,Scale=1}
\fi
\usepackage{lmodern}
\ifPDFTeX\else
  % xetex/luatex font selection
\fi
% Use upquote if available, for straight quotes in verbatim environments
\IfFileExists{upquote.sty}{\usepackage{upquote}}{}
\IfFileExists{microtype.sty}{% use microtype if available
  \usepackage[]{microtype}
  \UseMicrotypeSet[protrusion]{basicmath} % disable protrusion for tt fonts
}{}
\makeatletter
\@ifundefined{KOMAClassName}{% if non-KOMA class
  \IfFileExists{parskip.sty}{%
    \usepackage{parskip}
  }{% else
    \setlength{\parindent}{0pt}
    \setlength{\parskip}{6pt plus 2pt minus 1pt}}
}{% if KOMA class
  \KOMAoptions{parskip=half}}
\makeatother
\usepackage{graphicx}
\makeatletter
\newsavebox\pandoc@box
\newcommand*\pandocbounded[1]{% scales image to fit in text height/width
  \sbox\pandoc@box{#1}%
  \Gscale@div\@tempa{\textheight}{\dimexpr\ht\pandoc@box+\dp\pandoc@box\relax}%
  \Gscale@div\@tempb{\linewidth}{\wd\pandoc@box}%
  \ifdim\@tempb\p@<\@tempa\p@\let\@tempa\@tempb\fi% select the smaller of both
  \ifdim\@tempa\p@<\p@\scalebox{\@tempa}{\usebox\pandoc@box}%
  \else\usebox{\pandoc@box}%
  \fi%
}
% Set default figure placement to htbp
\def\fps@figure{htbp}
\makeatother
\setlength{\emergencystretch}{3em} % prevent overfull lines
\providecommand{\tightlist}{%
  \setlength{\itemsep}{0pt}\setlength{\parskip}{0pt}}
\usepackage{fontspec}
\usepackage{polyglossia}
\setmainlanguage{english}
\setotherlanguage{russian}
\usepackage{enumitem}
\setlist{nolistsep}
\usepackage{sectsty}
\sectionfont{\centering}
\usepackage[margin=1in]{geometry}
\usepackage{multicol}
\usepackage{bookmark}
\IfFileExists{xurl.sty}{\usepackage{xurl}}{} % add URL line breaks if available
\urlstyle{same}
\hypersetup{
  pdftitle={Karolina Nugumanova},
  colorlinks=true,
  linkcolor={Maroon},
  filecolor={Maroon},
  citecolor={Blue},
  urlcolor={blue},
  pdfcreator={LaTeX via pandoc}}

\title{Karolina Nugumanova}
\author{}
\date{\vspace{-2.5em}}

\begin{document}
\maketitle

Scuola Normale Superiore Address: Palazzo Strozzi, Florence, Italy
Emails: karolina.nugumanova@sns.it, kar.nugumanova@gmail.com Google
Scholar: \textless a href=
``\url{https://scholar.google.com/citations?user=JBryBecAAAAJ&hl=en}''
ORCID: 0000-0002-4811-0837 Webpage: karolina-nugumanova.github.io

\subsection{Research Agenda}\label{research-agenda}

My research examines how gender, emotions, and political opportunity
structures shape political participation and mobilization under
authoritarianism and in exile. I combine large-scale survey data, panel
studies, experiments, and qualitative interviews to study post-2022
Russian emigration, feminist anti-war activism, and emotional mechanisms
of sustained political engagement.

My work contributes to political behavior, migration studies, and gender
politics, with a particular focus on authoritarian contexts and
transnational activism.

\subsection{Academic Appointments}\label{academic-appointments}

\begin{itemize}
\tightlist
\item
  \textbf{Visiting Research Fellow}, \emph{Institute of Social Science},
  The University of Tokyo (upcoming 2026)\\
  Supervisor: Masaaki Higashijima
\item
  \textbf{Exchange Student}, \emph{Globalisation Program}, Venice
  International University (2021)
\end{itemize}

\subsection{Education}\label{education}

\begin{itemize}
\tightlist
\item
  \textbf{PhD in Political Science and Sociology} (expected defense
  2027), \emph{Scuola Normale Superiore}, Italy\\
  Supervisors: Martín Portos García; Lorenzo Zamponi\\
  Dissertation: \emph{Gendered Mobilization in Exile: Understanding the
  Activism of Post-2022 Russian Emigrants through Emotions, Structure,
  and Movement Dynamics}
\item
  \textbf{MA in Sociology}, \emph{European University at
  St.~Petersburg}, Russia (2020--2022)\\
  GPA: 4.6 / 5\\
\item
  \textbf{BA in Sociology}, \emph{Novosibirsk State University}, Russia
  (2016--2020)\\
  GPA: 4.3 / 5
\end{itemize}

\subsection{Peer-Reviewed Journal
Articles}\label{peer-reviewed-journal-articles}

Nugumanova, Karolina. \textbf{Micromobilising Emotion: How Feminist
Anti-War Resistance Builds Affective Infrastructure in Exile.}
\emph{Communist and Post-Communist Studies} (SSCI, Scopus), 2026.

\subsection{Manuscripts Under Review / In
Preparation}\label{manuscripts-under-review-in-preparation}

\begin{itemize}
\item
  Nugumanova, K., Kamalov, E., \& Sergeeva, I. \emph{Reverse Gender Gap
  in Transnational Activism: War-Driven Political Engagement Among
  Russian Post-2022 Emigrants.} (in preparation; target journals:
  \emph{Journal of Ethnic and Migration Studies})
\item
  Kamalov, E., Nugumanova, K., \& Sergeeva, I. \emph{Exposure to
  Transnational Repression and Political Engagement: Evidence from a
  Panel of Russian Exiles.} (in preparation; target journals:
  \emph{APSR/AJPS})
\end{itemize}

\subsection{Selected Works in
Progress}\label{selected-works-in-progress}

\begin{itemize}
\tightlist
\item
  \emph{Destination Preferences in Secondary Migration Among Politically
  Induced Migrants from Authoritarian States: Evidence from Conjoint
  Experiment}, with E. Kamalov and I. Sergeeva\\
\item
  \emph{The Gendered Opportunity Structure of Exile: How Host-Country
  Contexts Shape Migrant Political Participation}\\
\item
  \emph{Two Percent of Trust: Methodological Innovations for Retaining
  Fearful Respondents in Panel Surveys}, with E. Kamalov and I. Sergeeva
\item
  \emph{New Russian Migration to Latin America: Mobile Middle Classes,
  Flight from Illiberalism, and the Formation of New Urban Communities},
  with S. Ruseishvili, E. Kamalov, and I. Sergeeva
\end{itemize}

--\textgreater{}

--\textgreater{} --\textgreater{} --\textgreater{} --\textgreater{}
--\textgreater{} --\textgreater{}

\subsection{Academic Positions and Research Projects
(selected)}\label{academic-positions-and-research-projects-selected}

\begin{itemize}
\item
  \textbf{Principal Investigator},
  \href{https://readymag.website/u94255285/wardreams/2/}{\emph{The
  Science of Dreams}} (2022--present) --- National mixed-methods survey
  (N ≈ 1,500) examining war-related dream narratives among Russians
  during the full-scale invasion of Ukraine. Research cited in Russian
  and international media, including \emph{Meduza}, \emph{Republic},
  \emph{Paper}, \emph{Posle Media}, \emph{The Dial} and others.
\item
  \textbf{Co-Investigator}, \emph{OutRush Panel Survey} (2022--present)
  --- Large-scale longitudinal panel study of Russian political
  emigrants (N ≈ 25,000; 100+ countries), combining survey data,
  conjoint experiments, and qualitative modules. Funded by the European
  University Institute and the U.S. Russia Foundation. Research cited in
  100+ Russian and international media including \emph{The New York
  Times, Financial Times, Bloomberg, Al Jazeera, BBC, and Meduza}, among
  others.
\item
  \textbf{Co-Investigator}, \emph{A Comparative Study of Russian Migrant
  Communities following the Full-Scale Invasion of Ukraine (DemEx
  Project)} --- NSF-funded (2024--2026) multi-country mixed-methods
  project combining large-scale surveys, focus groups, and in-depth
  interviews with post-2022 Russian emigrants. Collaborative project
  between George Washington University and Carleton University, led by
  Marlene Laruelle et al.
\item
  \textbf{Co-Investigator}, \emph{A Study on Feminist Anti-War
  Resistance in Exile}(2023--2024) --- Qualitative interview-based study
  (60 in-depth interviews) with activists of the Feminist Anti-War
  Resistance movement abroad. Funded by the University of Oslo.
\item
  \textbf{Research Assistant}, \emph{Building a Commons in the Russian
  Diaspora} (2022--2023) --- Large-scale qualitative project based on
  580 in-depth interviews with post-2022 Russian migrants in Serbia,
  Kazakhstan, Kyrgyzstan, Armenia, Georgia, and Turkey.
\item
  \textbf{Expert Consultant / Foresight Group Facilitator}, \emph{EU
  Public Diplomacy toward Russia} --- European Commission--funded
  project (2024--2025). Provided expert analysis on Russian emigration
  and exile communities and facilitated foresight group discussions with
  Russian cultural workers in exile.
\end{itemize}

\subsection{Research Reports and Analytical
Publications}\label{research-reports-and-analytical-publications}

\begin{itemize}
\tightlist
\item
  \href{https://papers.ssrn.com/sol3/papers.cfm?abstract_id=5204018}{\emph{On
  the Move: Mobility, Integration, and Dynamics of Russian Emigration in
  2022--2024}}, with Emil Kamalov and Ivetta Sergeeva, 2025\\
\item
  \href{https://outrush.io/report_latam}{\emph{From Russia to Latin
  America: Integration, Challenges, and Aspirations of Russian Post-2022
  Emigrants}}, with Emil Kamalov and Ivetta Sergeeva, 2025\\
\item
  \href{https://www.research-collection.ethz.ch/handle/20.500.11850/692533}{\emph{Men's
  Words, Women's Work: Exploring the Reverse Gender Gap in Post-2022
  Russian Emigration}}. In \emph{Russian Analytical Digest (RAD)}, 2024
\item
  \href{https://outrush.io/report_artists_2024}{\emph{Mapping and
  Gathering Russian Artists and Cultural Workers in Exile}}, 2024\\
\item
  \href{https://outrush.io/report_january_2024}{\emph{One and a Half
  Years Later: Progress and Barriers in the Integration of Russian
  Emigrants}}, 2024\\
\item
  \href{https://thefebruaryjournal.org/index.php/tfj/article/view/67}{\emph{`Great,
  Great Sorrow and Eternal Silence': An Experiment in Sociological Dream
  Interpretation after the 24th of February 2022}}. In \emph{The
  February Journal}, 2023
\item
  \href{https://posle.media/language/en/how-do-you-sleep-during-the-war/}{\emph{How
  do you sleep during the war?}}, \emph{Posle Media}, 2023\\
\item
  \href{https://republic.ru/posts/107912}{\emph{I had a dream about
  Putin. What war-time dreams tell us about reality}}, \emph{Republic},
  2023 (in Russian)
\end{itemize}

\subsection{Awarded Grants and
Fellowships}\label{awarded-grants-and-fellowships}

\begin{itemize}
\item
  \textbf{Democracy in Exile (DemEx)} --- National Science Foundation
  (NSF), multi-year collaborative research grant, with Marlene Laruelle
  et al.~(2024--2026).
\item
  \textbf{Mapping Opportunities for Social Infrastructure of Russian
  Anti-War Migration in Latin America} --- U.S. Russia Foundation
  (USRF), competitive research grant, with Vladimir Rouvinski, Emil
  Kamalov, and Ivetta Sergeeva (2024).
\item
  \textbf{PhD Fellowship} --- PNRR (Italy), fully funded doctoral grant
  (2023--2027).
\item
  \textbf{Visiting Fellow Grant} --- Scuola Normale Superiore (2026).
\item
  \textbf{Mobility Research Grant} --- International mobility funding
  (2023-2025)
\item
  \textbf{MA Scholarship} --- European University at St.~Petersburg
  (2020--2022).
\item
  \textbf{Russian National Sociology Olympiad Grant} --- HSE \& Yandex,
  awarded twice (2021, 2022).
\end{itemize}

\subsection{Selected Academic
Conferences}\label{selected-academic-conferences}

\emph{(peer-reviewed international conferences)}

\subsubsection{2025}\label{section}

\begin{itemize}
\tightlist
\item
  \emph{European Consortium for Political Research (ECPR)} ---
  Thessaloniki --- Presenter\\
\item
  \emph{ASEEES Annual Convention} --- Washington, DC --- Roundtable
  Speaker\\
\item
  \emph{Gens Conference: International Conference on Gender Studies}
  (ISA / ESA / Italian Sociological Association) --- Stintino ---
  Presenter
\end{itemize}

\subsubsection{2024}\label{section-1}

\begin{itemize}
\tightlist
\item
  \emph{ASEEES Annual Convention} --- Boston --- Presenter\\
\item
  \emph{Aleksanteri Conference ``Resisting Authoritarianism in
  Eurasia''} --- University of Helsinki --- Presenter\\
\item
  \emph{International Conference ``(In)Visible Russian (Anti-)War
  Migration''} --- Warsaw --- Presenter
\end{itemize}

\subsubsection{2023}\label{section-2}

\begin{itemize}
\tightlist
\item
  \emph{Aleksanteri Conference ``Decolonizing Space in the Global
  East''} --- University of Helsinki --- Presenter\\
\item
  \emph{International Conference ``Problematizing Migration: Mobility
  and Vulnerabilization in an Age of Inequalities''} --- University of
  Palermo --- Presenter
\end{itemize}

\subsection{Workshops and Invited
Talks}\label{workshops-and-invited-talks}

\subsubsection{2025}\label{section-3}

\begin{itemize}
\tightlist
\item
  \emph{Gender, Peace and Security: Resistance and Resilience} --- BIEA
  Nairobi --- Paper Presenter\\
\item
  \emph{Mobility and Diversity: War-Induced Russian Migration since
  February 2022} --- ZOiS, Berlin --- Invited Paper Presenter\\
\item
  \emph{Russia's Outlook in the New Geopolitical Reality} --- Russia
  Expert Seminar, European University Institute --- Invited Presenter
\end{itemize}

\subsubsection{2024}\label{section-4}

\begin{itemize}
\tightlist
\item
  \emph{Migration Working Group} --- European University Institute ---
  Paper Presenter\\
\item
  \emph{LCSR International Workshop ``Recent Advances in Comparative
  Study of Values''} --- Moscow (online) --- Paper Presenter\\
\item
  \emph{Cross-Border Voices: Emigration, Gender, and Activism after the
  Russian Invasion of Ukraine} --- ZoiS, EDRAM Workshop, Ilia State
  University (Tbilisi) --- Invited Presenter\\
\item
  \emph{Cross-Border Repression and Solidarity among Russian Exiles} ---
  Riga --- Invited Presenter
\end{itemize}

\subsection{Academic Service}\label{academic-service}

\begin{itemize}
\item
  Ad hoc reviewer for \emph{Problems of Post-Communism} (SSCI-indexed
  journal).
\item
  Conference Discussant:

  \begin{itemize}
  \tightlist
  \item
    ASEEES Annual Convention --- Washington, DC, 2025.
  \item
    European Graduate Network (EGN) Conference --- Zurich, 2024.
  \end{itemize}
\end{itemize}

\subsection{Skills}\label{skills}

\begin{itemize}
\item
  \textbf{Languages:} Russian (native), English (advanced), Italian
  (intermediate)
\item
  \textbf{Methods:} multilevel modeling; panel data analysis; survey and
  conjoint experiments; difference-in-differences; Blinder--Oaxaca
  decomposition; qualitative interviews
\item
  \textbf{Software:} R, LaTeX, Qualtrics, Git, MaxQDA
\item
  \textbf{Research Practice:} reproducible research; open science;
  research data management (certified by Scuola Normale Superiore);
  survey design and panel studies; intersectional feminist methodology;
  ethical research with human subjects (CITI certification)
\end{itemize}

\begin{center}\rule{0.5\linewidth}{0.5pt}\end{center}

Updated: January 2026

\end{document}
